\documentclass[a4paper, 12pt]{report}

\usepackage[utf8]{inputenc}
\usepackage[T1]{fontenc}
\usepackage{lmodern}
\usepackage[ngerman]{babel}
\usepackage[margin=3.0cm,top=4cm]{geometry}
\usepackage[]{graphicx}

\usepackage{array}
\usepackage{lastpage}
\usepackage{blindtext} 
\usepackage{amsmath} 
\usepackage{nicefrac}
\usepackage{float}


\newcolumntype{L}[1]{>{\raggedright\let\newline\\\arraybackslash\hspace{0pt}}m{#1}}
\newcolumntype{C}[1]{>{\centering\let\newline\\\arraybackslash\hspace{0pt}}m{#1}}
\newcolumntype{R}[1]{>{\raggedleft\let\newline\\\arraybackslash\hspace{0pt}}m{#1}}


\usepackage{sectsty}
\sectionfont{\fontsize{12}{15}\selectfont}

\renewcommand{\familydefault}{\sfdefault}

\usepackage{fancyhdr}

\pagestyle{fancy}
\fancyhf{}
\fancyhead[L]{Tschebyscheff Tiefpass}
\fancyhead[R]{Stefan Urban}
\fancyfoot[R]{Seite \thepage\ von \pageref{LastPage}}
\renewcommand{\headrulewidth}{0.4pt}

%\addtolength{\parskip}{2mm}
\setlength{\parindent}{0pt}

\begin{document}

	Normierung erfolgt auf Durchlassfrequenz: $ \Omega = \frac{\omega}{\omega_p} $.
	\vspace{0.5cm}

    \begin{minipage}[t]{0.5\textwidth}
		\section*{Verstärkung}
	    	\[ H(\Omega) = \frac{1}{\sqrt{1 + c^2 \cdot T^2_n(\Omega)}} \]
    \end{minipage}
    \begin{minipage}[t]{0.5\textwidth}
		\section*{Dämpfung}
		    \[ a(\Omega) = 10dB \cdot \log{\left(1 + c^2 \cdot T^2_n(\Omega)\right)} \]
		   	\[ a_p = 10dB \cdot \log{\left(1+c^2\right)} \]
    \end{minipage}
	
	\vspace{-0.5cm}  
	
\section*{Asymtoten}
    \begin{minipage}[t]{0.5\textwidth}
		n ungerade:
		\begin{align*}
		 	a_{\Omega\rightarrow 0,n} &= 0dB &\\
		 	a_{\Omega\rightarrow \infty,n} &= 20dB \cdot \log{\left(2^{(n-1)} \cdot c\right)} &\\ &\qquad + n \cdot 20dB \cdot \log{(\Omega)} &
		\end{align*}
    \end{minipage}
    \begin{minipage}[t]{0.5\textwidth}
		n gerade:
		\begin{align*}
		 	a_{\Omega\rightarrow 0,n} &= a_p &\\
		 	a_{\Omega\rightarrow \infty,n} &= 20dB \cdot \log{\left(2^{(n-1)} \cdot c\right)} &\\ &\qquad + n \cdot 20dB \cdot \log{(\Omega)} &
		\end{align*}
    \end{minipage}
    
\vspace{0.8cm}
    
\begin{minipage}[t]{0.6\textwidth}
	\section*{Aufwandsabschätzung}
		\[ \frac{d}{c} = \sqrt{\frac{10^{a_s/10dB} - 1}{10^{a_p/10dB} - 1}} \qquad n \ge \frac{\cosh^{-1}{\left(\frac{d}{c}\right)}}{\cosh^{-1}{\left(\frac{\omega_s}{\omega_p}\right)}} \]

\end{minipage}
\begin{minipage}[t]{0.3\textwidth}
	\section*{Nullstellen}
		Alle n Nullstellen sind im Unendlichen.
\end{minipage}
   	
	
\section*{Pole}
	Zuerst Butterworth Polwinkel \\ ausrechnen:		
	\[ \varTheta_{xBk} = \pi \cdot \left( \frac{2k-1}{2n} \right)  \]
	
	Kreisradien ausrechnen:
	
	\begin{minipage}[t]{0.33\textwidth}
		\vspace{-0.5cm}   	
		\[ \ln{(\alpha)} = \sinh^{-1}{\left(\frac{1}{c}\right)} \]
	\end{minipage}
   	\begin{minipage}[t]{0.33\textwidth}
		\[ R_i = \sinh{\left(\frac{\ln{\alpha}}{n}\right)} \]
   	\end{minipage}
   	\begin{minipage}[t]{0.33\textwidth}
		\[ R_a = \cosh{\left(\frac{\ln{\alpha}}{n}\right)} \]
   	\end{minipage}

	\vspace{0.3cm}   	
	Kreisradien auf Durchlassfrequenz normieren:
	
	\begin{minipage}[t]{0.5\textwidth}
		\[ r_i = R_i \cdot \omega_p \]
	\end{minipage}
   	\begin{minipage}[t]{0.5\textwidth}
		\[ r_a = R_a \cdot \omega_p \]
   	\end{minipage}
   	
   	\vspace{0.3cm}
   	Pole berechnen:
   	
	\begin{minipage}[t]{0.6\textwidth}
		\[ \omega_{xTk} = \sqrt{\left(r_i \cdot \sin{\varTheta_{xBk}}\right)^2 + \left(r_a \cdot \cos{\varTheta_{xBk}}\right)^2} \]
		\[ \varTheta_{xTk} = \sin^{-1}{\left(\frac{1}{2Q}\right)} \]
	\end{minipage}
   	\begin{minipage}[t]{0.4\textwidth}
   		\vspace{-0.3cm}
   		\begin{align*}
   			Q_{xTk} &= \frac{1}{2} \cdot \sqrt{1+ \left(\frac{r_a / r_i}{\tan{\varTheta_{xBk}}}\right)^2} &\\
   			&= \frac{f_x}{2 \cdot r_i \cdot \sin{\varTheta_{xBk}}} &
   		\end{align*}

   	\end{minipage}
   	
   	 	
\clearpage

   	
\end{document}
